\chapter{Simple Harmonic Motion}

\section{Introduction: Periodic Motion}

There are two basic ways to measure time: by duration or periodic motion. Early clocks measured duration by calibrating the burning of incense or wax, or the flow of water or sand from a container. Our calendar consists of years determined by the motion of the sun; months determined by the motion of the moon; days by the rotation of the earth; hours by the motion of cyclic motion of gear trains; and seconds by the oscillations of springs or pendulums. In modern times a second is defined by a specific number of vibrations of radiation, corresponding to the transition between the two hyperfine levels of the ground state of the cesium 133 atom.


Sundials calibrate the motion of the sun through the sky, including seasonal
corrections. A clock escapement is a device that can transform continuous movement into discrete movements of a gear train. The early escapements used oscillatory motion to stop and start the turning of a weight-driven rotating drum. Soon, complicated escapements were regulated by pendulums, the theory of which was first developed by the physicist Christian Huygens in the mid 17th century. The accuracy of clocks was increased and the size reduced by the discovery of the oscillatory properties of springs by Robert Hooke. By the middle of the 18th century, the technology of timekeeping advanced to the point that William Harrison developed timekeeping devices that were accurate to one second in a century. 

\subsection{Simple Harmonic Motion}
One of the most important examples of periodic motion is simple harmonic motion (SHM), in which some physical quantity varies sinusoidally. Suppose a function of time has the form of a sine wave function,

\begin{sagesilent}
var('t,T,A')
p = var('pi')
y = function('y',t)
z = A*sin(2*p*t/T)
\end{sagesilent}

\begin{equation}
\sage{y} = \sage{z}
\end{equation}

where $A > 0$ is the amplitude (maximum value). The function $y(t)$ varies between $A$ and $− A$, because a sine function varies between +1 and −1. A plot of $y(t)$ vs. time is shown in Figure 1.1

\begin{figure}[h!]
$$	\includegraphics[scale=0.4]{graph.png}$$
	\caption{Sinusoidal function of time}
\end{figure}

The sine function is periodic in time. This means that the value of the function at time t will be exactly the same at a later time $t′ = t + T$ , where $T$ is the period. The \textbf{frequency}, $f$, is defined to be

\begin{sagesilent}
var('T')
z = 1/T
\end{sagesilent}

\begin{equation}
f = \sage{z}
\end{equation}

The SI unit of frequency is inverse seconds, hertz [Hz]. The angular frequency of oscillation is defined to be 

\begin{sagesilent}
var('omega,T,pi,f')
z=2*pi/T
z1=2*pi*f
\end{sagesilent}

\begin{equation}
\sage{omega} = \sage{z} = \sage{z1}
\end{equation}

and is measured in radians per second. We therefore have several different mathematical representations for sinusoidal motion

\begin{sagesilent}
var('A,pi,T,t,omega,f')
y = function('y',t)
z=A*sin(2*pi*t/T)
z1=A*sin(2*pi*f*t)
z2=A*sin(omega*t)
\end{sagesilent}

\begin{equation}
\sage{y} = \sage{z} = \sage{z1} = \sage{z2}
\end{equation}

\section{Simple Harmonic Motion: Analytic approach}
Our first example of a system that demonstrates simple harmonic motion is a spring object system on a frictionless surface, shown in Figure 23.2

\begin{figure}[h!]
$$\includegraphics[scale=0.25]{diagram.png}$$
\caption{Spring-object system}
\end{figure}

The object is attached to one end of a spring. The other end of the spring is attached to a wall at the left in Figure 1.2. Assume that the object undergoes one-dimensional motion. The spring has a spring constant $k$ and equilibrium length $l_{eq}$ . Choose the origin at the equilibrium position and choose the positive $x$ -direction to the right in the Figure 1.2. In the figure, $x > 0$ corresponds to an extended spring, and $x < 0$ to a compressed spring. Define $x(t)$ to be the position of the object with respect to the equilibrium position. The force acting on the spring is a linear restoring force, $F_{x} = −k x$ (Figure 1.3). The initial conditions are as follows. The spring is initially stretched a distance $l_{0}$ and given some initial speed $v_{0}$ to the right away from the equilibrium position. The initial position of the stretched spring from the equilibrium position (our choice of origin) is $x_{0} = (l_{0}-l_{eq} ) > 0$ and its initial $x$ -component of the velocity is $v_{x,0} = v_{0} > 0$. 

\begin{figure}[h!]
$$\includegraphics[scale=0.25]{diagram1.png}$$
\caption{Free-body force diagram for spring-object system}
\end{figure}

Newton’s Second law in the $x$ -direction becomes

\begin{sagesilent}
var('k,t,m')
x = function('x',t)
z=-k*x
\end{sagesilent}

\begin{equation}
\sage{z} =  m\frac{\mathrm{d}^{2}x}{\mathrm{d}t^{2}}
\end{equation}

This equation of motion, Eq. (1.5), is called the \textbf{simple harmonic oscillator equation} (SHO). Because the spring force depends on the distance $x$ , the acceleration is not constant. Eq. (1.5) is a second order linear differential equation, in which the second derivative of the dependent variable is proportional to the negative of the dependent variable,

\begin{sagesilent}
z = -k*x/m
\end{sagesilent}

\begin{equation}
\frac{\mathrm{d}^{2}x}{\mathrm{d}t^{2}} = \sage{z}
\end{equation}

In this case, the constant of proportionality is $k/m$,

%\section{Derivation}
\begin{sagesilent}
# Declaring spring constant, mass, time as k, m, t respectively in SageMath
k=var('k')
m=var('m')
t=var('t')
x=function('x',t)

# Assuming spring constant and mass greater than zero
assume(k>0)
assume(m>0)

# Writing our differential equation in SageMath
de = diff(x,t,2)+(k*x)/m

# Soloving our differential equation in SageMath
z=desolve(de,dvar=x,ivar=t)
\end{sagesilent}

% PRINTING OUR DIFFERENTIAL EQUATION
\begin{equation}
  \frac{\mathrm{d}^{2}x}{\mathrm{d}t^{2}} +  \sage{k*x/m} = 0
\end{equation}

Since mass and spring constant cannot be negative and zero so, solving the above differential equation, we get,
% PRINTING A SOLUTION OF OUR DIFFERNTIAL EQUATION WHICH IS SOLOVED BY SAGEMATH

\begin{equation}
x = \sage{z}
\end{equation}

\begin{sagesilent}
# Declaring mass and spring constant equal to 1
m=1
k=1
de = diff(x,t,2)+k*x/m

# Soloving our differential equation by setting up the initial or boundary conditions.
z2 = desolve(de,dvar=x,ivar=t,ics=[0,1,0])
# Plotting the graph of solution of our differential equation
pl1 = plot(z2,(t,0,10),color=('red'), figsize = (4,2.5),axes_labels=['$t$','$x$'], fontsize=7)

m=1
k=1
z3 = desolve(de,dvar=x,ivar=t,ics=[1,1,0])
print z3
pl2 = plot(z3,(t,0,10), color=('green'))

m=1
k=1
z4 = desolve(de,dvar=x,ivar=t, ics=[0,0,1])
pl3 = plot(z4,(t,0,10),color=('yellow'))

m=1
k=1
z5 = desolve(de,dvar=x,ivar=t, ics=[1,1,1])
pl4 = plot(z5,(t,0,10),color=('violet'))
\end{sagesilent}

For $m=1$ and $k=1$, then the graph is shown by Figure 1.4
% ADDING CAPTION TO GRAPH
\begin{figure}[h!]
% PRINTING GRAPH BY USING SAGEMATH
$$\sageplot{pl1+pl2+pl3+pl4}$$
\caption{Graph}
\end{figure}

red, green, yellow and violet curves represents the initial or boundary conditions [0, 1, 0], [1, 1, 0], [0, 0, 1] and [1, 1, 1] respectively. Putting these initial conditions in Eq. (1.8) one by one. Here $t$ corresponds to first term, $x$ corresponds to second term and $\frac{\mathrm{d}x}{\mathrm{d}t}$ corresponds to third term in the initial conditions.   

